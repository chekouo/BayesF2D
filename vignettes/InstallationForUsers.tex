%\VignetteIndexEntry{Installation for Users}
\documentclass{article}
\usepackage[utf8]{inputenc}
\usepackage{hyperref}
\hypersetup{colorlinks=true}
\usepackage{menukeys} 
\usepackage{graphicx}
\usepackage{float}
\usepackage{listings,textcomp} % for R code blocks

\title{User Installation Instructions Manual for \texttt{BayesF2D}}
\author{Thierry Chekouo and Shariq Mohammed}
\date{\today}

\usepackage{Sweave}
\begin{document}
\Sconcordance{concordance:InstallationForUsers.tex:InstallationForUsers.Rnw:%
1 14 1 1 0 109 1 1 2 1 0 6 1 1 7 6 0 1 2 5 1 1 3 2 0 1 3 5 0 1 2 3 1}


\maketitle
The \texttt{BayesF2D} software is a package for R written in R and C.  It has utilities which allow users to associate Functional 2D objects (i.e GLCM matrices) with scalar responses, and predict the later.  Various functions in R are called by the user to input  data information. The main R function, \texttt{BayesF2DMCMC}, estimates our parameters using MCMC techniques. This function calls a C function based on the user's input, and is compiled and shared with the rest of the \texttt{BayesF2D} C code in order to estimate the parameters of the model.



\section{\Large Instructions for Installing and Using \texttt{BayesF2D} on a Windows Computer}
Because the \texttt{BayesF2D} package compiles C code in response to user input, more setup is required for the \texttt{BayesF2D} package than for many others. The general requirements are as follows:

\begin{enumerate}
\item R must be installed. 
\item Rtools must be installed so that C code can be compiled on Windows. See Section \ref{sec:Rtools}.
\item Additional GSL libraries must be installed so that the C code can use GSL for matrix multiplication routines and random number generations. See Section \ref{sec:gsl}.
\item The environment variable for the system Path must contain Rtools. See Section \ref{sec:Rtools}.
\item An environment variable for GSL called LIB\_GSL must be created. See Section \ref{sec:libgsl}.
\end{enumerate}

Each of these steps is detailed below.  %We acknowledge that this additional setup can be bothersome, but we believe the ease of use for the rest of the package and the wide variety of models it is possible to fit with it will compensate for this initial burden.  Hopefully you will agree!

\subsection{\Large Instructions for Installation of R}
\label{sec:R}
\begin{itemize}
    \item Make sure that R is installed and Check the directory in which R is installed.
    \item If not go to \href{https://www.r-project.org/}{https://www.r-project.org/}  and click on download R and select any mirror.
\end{itemize}


\subsection{\Large Instructions for Installation of R-tools }
\label{sec:Rtools}
\begin{enumerate}
\item If you already have R-tools on your machine, make sure the version of R-tools matches with your R version. This package was tested with the R-tools version 3.5. Newer versions than 3.0 may work fine with \texttt{BayesF2D}.
\item Install R-tools through \url{https://cran.r-project.org/bin/windows/Rtools/}. Install the latest version of Rtools.
\item After saving the file ``Rtools34.exe'', double-click on the icon for the file to run it.
\item You will be asked what language to install it in - choose English.
\item The Rtools Setup Wizard will appear in a window. Click ``Next'' at the bottom of the R Setup wizard window.
\item The next page says ``Information'' at the top. Click ``Next'' again.
\item The next page says ``Select Destination Location'' at the top. By default, it will suggest to install R-tools in ``C:\textbackslash Rtools'' on your computer. You may also install R-tools in  {\em any other directory where there are {\bf no spaces} in the words describing the directory}. Click ``Next'' at the bottom of the R tools Setup wizard window.
\item The next page says ``Select components'' at the top. Make sure that the Cygwin Dlls box {\bf is checked}. Click ``Next'' again.
\item The next page says ``Select additional tasks'' at the top. Check the box to edit the system PATH. Click ``Next'' again.
\item Add the directory path containing  R Rtools. For instance,  add the path \texttt{C:\textbackslash Rtools\textbackslash bin;}  to your list of {\em system path} variable if not available .\\
%\texttt{C:\textbackslash Rtools\textbackslash bin;}
\item (optional) You may want to add also the path containing R.exe (e.g., in C:\textbackslash R\textbackslash R-3.5.1\textbackslash bin;) if you intend to install the \texttt{BayesF2D} with the command R CMD INSTALL from the command Windows.
\item The next page says ``Ready to install'' at the top. Click ``install''.
\item Rtools should now be installed. This will take about a minute. Click ``Finish''.
\end{enumerate}


\subsection{\Large Instructions for Installation of GSL}
\label{sec:gsl}
\begin{enumerate}
\item If you already have GSL libraries installed on your machine, you may skip this step and go to Section \ref{sec:libgsl}. But we advise to follow these steps. 
\item To install GSL libraries, go to \href{http://www.stats.ox.ac.uk/pub/Rtools/libs.html}{http://www.stats.ox.ac.uk/pub/Rtools/libs.html}.
\item Download ``local323.zip'' (or the latest version) by clicking it.  The 323 refers to R 3.2.3 but this installation works for 3.5.1.
\item Extract it into a new folder and copy and paste the extracted new folder into the  directory a``C:\textbackslash'' for instance (C:\textbackslash local323). 
\end{enumerate}



\subsection{\Large Setting up the GSL Environment Variable}
\label{sec:libgsl}
\begin{enumerate}
 \item Open - Control Panel\textbackslash System and Security\textbackslash System
 \item Click on Advanced system settings and then click on ``Environment Variables''
 \item Add a new {\em system} variable by clicking on New.  Note that this should be a {\em System} environment variable, not a {\em User} environment variable.
 \item Name the new variable as LIB\_GSL and set the variable value to the 
 directory containing the local323 files downloaded in Section \ref{sec:gsl} or the directory where your GSL libraries are installed. Example {C:/local323} but please note that the direction of these slashes is important. This slash / will work but {\em not} this one \textbackslash.
\end{enumerate}



\subsection{\Large Wrapping up the Installation Procedure for Windows}
\label{wrapup}
\begin{enumerate}
\item Open RGui, Rstudio or whatever editor you use to run R. Please type the following code to check whether the gsl commands can be found correctly:\\ \texttt{shell("echo \%LIB\_GSL\%")}\\ If this returns something like ``C:/local323'', then everything worked fine.
 \item If that command returns something like ``\%LIB\_GSL\%'', then something might be wrong with the GSL installation (Return to Section \ref{sec:gsl}) or GSL path (Return to Section \ref{sec:libgsl}).
 \item If the command worked fine, follow the steps in section \ref{sec:github} to finish the installation process in R.
\end{enumerate}



\section{\Large Instructions for Installing and Using \texttt{BayesF2D} On Mac}

\subsection{\Large Instructions for Installing Prerequisites on Mac}
\label{sec:macpre}
\begin{enumerate}
\item Install \texttt{Xcode} with the command line tools from your Apple store. For this and all the subsequent steps, please read the installation instructions on the individual websites carefully to pick the correct software versions for your operating system.
\item Install Homebrew. See this website for more details: \url{https://brew.sh}
\item Open the terminal window. In the terminal window install the gsl library by typing: ``brew install gsl''.
\item Follow the steps in section \ref{sec:maccheck} to verify that these steps worked properly.
\end{enumerate}

\subsection{\Large Checking the Installation for Mac}
\label{sec:maccheck}
\begin{enumerate}
\item Open RGui, Rstudio or whatever editor you use to run R. Please type the following code to check whether the gsl commands can be found correctly:\\ \texttt{system("gsl-config --cflags", intern=TRUE)}\\ When the command can not be found, and you know where it is stored (e.g., "/opt/local/bin"), we could then set the PATH variable by typing: \\
\texttt{Sys.setenv(PATH=paste0(Sys.getenv("PATH"),":","/opt/local/bin"))}\\ and then check again.
\item If the above failed, then something went wrong with one or several of the steps in Section \ref{sec:macpre}. Please go back and try repeating or checking that section.
\item Follow the steps in section \ref{sec:github} to finish the installation process in R.
\end{enumerate}

\section{\Large Getting \texttt{BayesF2D} from GitHub}
\label{sec:github}
\begin{enumerate}
\item Open RGui, Rstudio, or whatever editor you use to run R.
\item Install the package \texttt{devtools} from CRAN by typing install.packages(`\texttt{devtools}') and then library(devtools).
\item Install \texttt{BayesF2D} from GitHub as usual by typing install\_github(`\texttt{chekouo/BayesF2D}') and then library(`\texttt{BayesF2D}').
\item To test if \texttt{BayesF2D} is installed correctly, run this  example:\\

\begin{Schunk}
\begin{Sinput}
> library(BayesF2D)
> data(nonsymCor)
> Cor = nonsymCor
> s.cov0 = 5*matrix(c(1,-0.1,-0.1,1), nrow=2)
> S0 = list(s.cov0, s.cov0, s.cov0, s.cov0)
> s.cov1 = 10*matrix(c(1,-0.1,-0.1,1), nrow=2)
> S1 = list(s.cov1, s.cov1, s.cov1, s.cov1)
> glcm = BayesF2D::simulateGLCM_binary(n0=30,n1=20,GLCM.type = "nonsym",
+                                         m0 = list(c(1,7),c(3,5),c(5,3),c(7,1)),
+                                         S0 = S0,
+                                         m1 = list(c(2,6),c(2,2),c(5,2),c(7,1)),
+                                         S1 = S1,
+                                         Gam = Cor,
+                                         noise = 3)
> meth1="NoCorBeta"
> y=glcm$y ## binary response
> N=length(y)
> R=4# number of imaging sequences
> TT=8 # number of image intensities
> X=array(0,dim=c(R,N,TT,TT))
> for (i in 1:N){
+ X[,i,,]=glcm$GLCM[[i]];
+ }
> Res<-BayesF2DMCMC(covbCov=(meth1=="CorBeta"),y=y,typeoutcome="binary",
+                   TwoDX=X,nbriter=5000,nbrburnin=1000,chainNber=1,
+                   hypsigm_rr=c(0.5,0.5),h=1)
\end{Sinput}
\end{Schunk}
\item If you're on Windows and everything worked fine until you tried to run the model, something probably went wrong with installing Rtools (Section \ref{sec:Rtools}) or installing GSL (Section \ref{sec:gsl}). Please refer to those sections for troubleshooting.
\end{enumerate}

\end{document}
